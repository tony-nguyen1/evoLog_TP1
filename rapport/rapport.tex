% Une ligne commentaire débute par le caractère « % »

\documentclass[a4paper]{article}

% Options possibles : 10pt, 11pt, 12pt (taille de la fonte)
%                     oneside, twoside (recto simple, recto-verso)
%                     draft, final (stade de développement)

\usepackage[utf8]{inputenc}   % LaTeX, comprends les accents !
\usepackage[T1]{fontenc}      % Police contenant les caractères français
\usepackage[francais]{babel}
\usepackage{fullpage}
\usepackage{multicol}
\usepackage{hyperref}
\hypersetup{
    colorlinks=true,
    linkcolor=blue,
    filecolor=magenta,      
    urlcolor=red
    }
\usepackage{bookmark}
\usepackage{blindtext}
\setlength\columnsep{30pt}



\usepackage{graphicx}  % pour inclure des images
\graphicspath{ {rapport/img/} }

%\pagestyle{headings}        % Pour mettre des entêtes avec les titres
                              % des sections en haut de page

\title{  TP1 : Analyse statique\\Évolution et restructuration des logiciels}
\author{Mohamad Satea Almallouhi - Tony Nguyen\\\emph{M1 Génie Logiciel}\\Faculté des Sciences\\Université de Montpellier.}
\date{6 octobre 2024}



\begin{document}
    \maketitle
    \begin{center}
        % \includegraphics[height=.95\textwidth]{power}
    \end{center}

    \begin{abstract}     % Résumé du travail
      \emph{Rapport d'exercice sur l'analyse statique}
    \end{abstract}
    \newpage
    %\dominitoc  % initializer les minitoc
    \tableofcontents
    \section*{Introduction}
            \addcontentsline{toc}{section}{Introduction}
            \paragraph{}
                Dans le cadre de l'Unité d'Enseignement Évolution et restructuration des logiciels, nous allons analysés un programme en observant son code source de manière statique.

                Tout d'abord, l'approche consistera a extraire un modèle du code source. Cela prendra la forme d'un AST.

                Par la suite, à l'aide du Patron de Conception Visiteur, nous allons parcourir l'arbre pour en extraire des propriétés.
        \section*{Démonstration vidéo}
            \addcontentsline{toc}{section}{Démonstration}
            \paragraph{}
                En ligne sur Youtube, à l'adresse URL \url{https://youtu.be/OuzuGsB9IC4} une démonstration vidéo de notre travail.
        \section*{Installation}
            \addcontentsline{toc}{section}{Installation}
            \paragraph{}
                Vous trouvez les instructions dans le README.md

    \newpage
    \begin{multicols}{2}
        [
            % To Do
            % \begin{itemize}
            %     \item diagram use case DONE
            %     \item diagram sequence for synchronisation DONE ~~~
            %     \item diagram state machine w/ tikz library to describe save function NO
            %     \item add code picture DONE
            %     \item add smartphone picture DONE
            %     \item !!! make an .apk file for easy install, no compilation !!! NO NEED, ALREADY DONE
            %     \item diagram class observer DONE
            %     \item diagram class observer specific (Fragment Manager) KO
            %     \item → !!! → vidéo ← !!! ← 
            % \end{itemize}
        %     Faire une vidéo, le rapport avec des screenshot des résultats et du code et enfin un read.md(instruction). En plus, pour le bonus, faire une belle application, des tests unitaires, utiliser Kotlin, faire le rapport en Latex.
        ]
        \section{Extraction de l'AST}
        \paragraph{} Cette partie est prise en charge par la classe ASTParser.
        \section{Implementation du visiteur}
        \paragraph{} Afin de parcourir l'AST, nous allons utilisé un visiteur. Pour en implémenter une compatible avec ASTParser, nous spécialisons la classe abstraite ASTVisitor.

        Pour pouvoir naviguer précisement dans l'arbre, il nous faut masquer la méthode visit(T node), avec T extends ASTNode. C'est en choisisant soignement le type T que nous pouvons atteindre l'informations recherché.

        \subsection{Le nombre de classe}
            % \paragraph{}
            %     Un fragment est une portion réutilisable de l'interface utilisateur de notre application. Ils sont utiles car ils sont modulaires et réutilisables. 
            % \paragraph{}
            %     Dans cette section, nous allons voir comment les créer, les afficher et comment les faire communiquer entre eux.
            % \subsection{Création}
            %     \paragraph{}
            %         Afin de créer un nouveau fragment, il nous suffit d'hériter de la classe de base Android : "\emph{Fragment}".

            %         Pour spécifier le comportement du nouveau fragment, il nous suffira de masquer la méthode \textbf{onCreateView()}.

            %         Dans ce TP, nous avons créer 2 fragments, \textbf{UserInputFragment} et \textbf{DisplayFragment}.
            % \subsection{Ajout}
            %     \noindent\includegraphics[width=.23\textwidth]{fragment/no_input}
            %     \noindent\includegraphics[width=.23\textwidth]{fragment/input}
            %     \paragraph{}
            %         Pour pouvoir les afficher, nous avons créer une activity et son layout vide.
                    
            %         Nous avons ensuite, ajouté une balise \textbf{fragment} pour chaque fragment crée. Il faut remarquer 2 attributs important.

            %         L'attribut \textbf{android:name} associe cette balise .xml à aux classe fragment crée dans la sous-section précédente.

            %         L'attribut \textbf{tools:layout} associe cette balise au layout indiqué.

            %         Nous savons à présent comment créer un fragment, et l'afficher.
            %         \begin{center}
            %             \noindent\includegraphics[width=.40\textwidth]{fragment/layout}
            %         \end{center}
            % %         tag fragment - attribut name pour la classe et attribut layout pour la vue
            % %         \\
            % %         Start\_toStartOf
            % %         Top\_toTopOf 
            % %         ...
            
            % % \subsection{Couper l'écran en 2}
            % %     C'était trop dur.
            % \subsection{Communication entre fragment}
            %     \paragraph{}
            %         Avant la communication, il est nécessaire de transmettre notre \emph{ModelView} contenant les différentes MutableLiveData.
            %     \paragraph{}
            %         Une fois le model placé dans le \emph{Bundle}, nous pouvons envoyer le message. Pour faire cela, nous utilisons la méthode \textbf{setFragmentResult(String,Bundle)} pour envoyer les données (contenus dans le bundle avec requestKey une chaine de charactères fixés à l'avance par le programmeur, le fragment receveur devra utiliser la même requestKey). Le message est maintenant envoyé, nous allons voir comment le recevoir.
            %         % \begin{center}
            %             \\\\
            %             \noindent\includegraphics[width=.47\textwidth]{fragment/set}
            %         % \end{center}
            %     \paragraph{}
            %         Nous allons maintenant voir comment recevoir le message.

            %         Il sera nécessaire d'utiliser la méthode \textbf{setFragmentResultListener()} de la classe \emph{FragmentManager}.

            %         L'argument listener sera une classe anonyme qui hérite de \emph{FragmentResultListener} dans laquelle on masque la méthode \textbf{onFragmentResult()}. Si nous avons bien utilisé les même requestKey, alors l'argument bundle contiendra les bonnes données. À partir de là, nos fragments possède la même instance du model.

            %         \noindent\includegraphics[width=.47\textwidth]{fragment/fragmentListen}
            % \subsection{La synchronisation et mise à jour}
            %     \noindent\includegraphics[width=.47\textwidth]{fragment/submittedSmall}
            %     \noindent\includegraphics[width=.47\textwidth]{fragment/syncSmall}
            %     \paragraph{}
            %         Lors de la détection d'un changements, nous souhaitons publié le changement et mettre à jour la vue. Pour cela, nous allons accéder aux différents \emph{MutableLiveData} dans notre \emph{ModelView} : \emph{UserInputViewModel}. Une fois fait, nous publions simplement la nouvelle valeur avec la méthode \textbf{postValue()}. Nous envoyons les valeurs lorsque l'utilisateur appuie sur "Submit" ou après qu'il ait selectioné l'option "Synchronise" puis modifie le texte.
            %         \\\\
            %         \noindent\includegraphics[width=.47\textwidth]{fragment/post}
            %     \paragraph{}
            %         Afin d'être notifié des plublications, il ne faut pas oublier de s'abboner aux MutableLiveData et les \textbf{observer}.
            %         \\\\
            %         \noindent\includegraphics[width=.47\textwidth]{fragment/observe}
            %     \paragraph{}
            %         Quand l'observateur est notifié, la méthode \textbf{onChanged()} est appelé et la nouvelle valeur est placé dans une \emph{<TextView>}.
            %         \\\\
            %         \noindent\includegraphics[width=.47\textwidth]{fragment/onChanged}
            % \subsection{Observer}
            %     % To Do
            %     % \begin{itemize}
            %     %     \item diagram class observer
            %     %     \item diagram class observer specific (Fragment Manager)
            %     % \end{itemize}
            %     \paragraph{}
            %         Voici un rappel du diagramme de classe du patron de conception \emph{Observateur} 
            %         \includegraphics[width=.49\textwidth]{observer}
            %     \paragraph{}
            %         Il faut remarquer que ce design pattern est essentiel dans ce tp pour implémenter la communication entre les fragments. On y retrouve les méthodes de souscriptions, notifications et mises à jour.

            %         On identifie les méthodes de notifications comme \textbf{setFragmentResult()} et \textbf{postValue()}.

            %         Les méthodes de mises à jour correspondendt à \textbf{setFragmentResultListener()} et \textbf{observe()}.
    %     \section{Persistance}
    %         \paragraph{}
    %             Nous allons maintenant nous intéressé à la conservation des données.
    %         \subsection{Sauvegarde}
    %             \noindent\includegraphics[width=.47\textwidth]{persistance/savedSmall}
    %             \noindent\includegraphics[width=.47\textwidth]{persistance/jsonFile}
    %             Lors d'un clique sur le boutton \emph{"Save"}, une écriture sur le stockage local du smartphone est enclenché.

    %             La méthode \textbf{saveJsonToFile()} se chargera de récupérer les informations intéressantes, les convertire en format JSON et d'écrire dans le fichier \emph{test.json}.
    %             \\\\\noindent\includegraphics[width=.47\textwidth]{persistance/write}
    %         \subsection{(Re-)Saisie automatique}
    %             \paragraph{}
    %                 Imaginons un petit scénario. 

    %                 Un utilisateur commence à saisir ses informations et fait une mauvaise manipulation. Il a appuyé sur le boutton de retour à l'écran d'accueil (Android) du smartphone. 

    %                 Quand il revient sur notre application, on aimerai qu'il ne recommence pas du début.
    %             \paragraph{Échec}
    %                 Malheuresement, nous n'avons pas réussi à implémenter cette fonctionnalités.

    %                 Dans le figure ci-dessous le code ne produisant \emph{pas} une sauvegarde de l'état du formulaire quand l'utilisateur quitte temporairement l'application.
    %                 \\\\\noindent\includegraphics[width=.47\textwidth]{persistance/on}
    %     \section{Réseau}
    %         Cet exerice n'a pas été réalisé.
    %     \section{Service}
    %         \noindent\includegraphics[width=.47\textwidth]{load/loaded}
    %         \paragraph{}
    %             Après qu'une écriture sur le stockage du téléphone a été faite, il nous est possible de récupérer le contenu de ce fichier et de l'afficher à l'utilisateur.

    %             À l'appuie du boutton "load", le service est démaré. \emph{FileDownloadService} va simplement lire le fichier de sauvegarde prédéfini \emph{test.json}.

    %             Il va ensuite, diffuser les valeurs à travers un le \emph{LocalBroadcastManager}.
    %         \paragraph{}
    %             Dans le DisplayFragment, Nous nous mettons à l'écoute du broadcast avec la fonction \textbf{registerReceiver()}.

    %             À la reception du broadcast, le text est écrasé par les différentes valeurs contenus dans le fichier grâce à la fonction \textbf{onReceive()}.
                
    %             On remarquera que l'\emph{Intent} et l'\emph{IntentFilter} ont le même nom.
    %             \\\\
    %             \noindent\includegraphics[width=.47\textwidth]{load/FileDownloadService}
    %             \noindent\includegraphics[width=.47\textwidth]{load/broadcastReceive}
    %     \section{Conclusion}
    %         \paragraph{}
    %             Pour conclure cet exercice, nous avons réaliser une petite application formulaire avec des fragments communiquant entre eux.
    %         \paragraph{}
    %             Malgré un sujet assez obscure, nous avons interpréter aux mieux de nos capacités. Nous avons réussi à implémenter, nous pensons, la majorité des fonctionnalités attendus : l'utilisateur peut afficher et sauvegarder ces données (même s'il n'a encore rien écrit), modifier l'affichage à la volée mais aussi les charger depuis un fichier.
    %     \begin{figure*}
    %         \centering
    %         \caption{Diagrame d'utilisation de notre application mobile}
    %         \includegraphics[width=.8\textwidth]{jpg/useCase}
    %         \label{fig:useCase}
    %     \end{figure*}
    %     \begin{figure*}
    %         \centering
    %         \caption{Diagrame de séquence \emph{possible} représentant la mise en place de la communication inter-fragment}
    %         \includegraphics[width=.8\textwidth]{jpg/sequence}
    %         \label{fig:useCase}
    %     \end{figure*}
    \end{multicols}
\end{document}
Quand on quitte l'appli, onStop et onSaveInstanceState sont invoqués.
Quand on fait retour, onStop est invoqué.
Quand on revient, onStart et onViewCreated sont invoqués.